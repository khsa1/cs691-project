\begin{abstract}

Frequent Subgraph Mining (FSM) is the general computational problem of
identifying what subgraphs occur frequently in a collection of graphs.
The dominant FSM algorithm over the past decade has been gSpan.
gSpan has been shown to give good performance relative to other existing
systems from both a computational time and memory consumption standpoint.
gSpan and algorithms like it are difficult to parallelize as it does a
recursive depth-first search of the solution space, and each search
can take a variable amount of time.

This paper demonstrates two parallel implementations of this algorithm that
use MPI to divide up the search space among multiple nodes within a cluster to
provide faster overall search times. We generally observe reasonable speedup
on small numbers of nodes but not on larger numbers of nodes. However, we
observe that the parallel performance of our method is dependent on the
dataset used. The limitations of these methods are explained and we suggest
a third implementation that should outperform our two current methods by
more evenly dividing the work, regardless of the structure of our dataset.

\end{abstract}
