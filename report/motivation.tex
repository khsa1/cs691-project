\section{Motivation}
\label{sec:motivation}

Frequent Subgraph Mining (FSM) is the general computational problem of 
identifying what subgraphs occur frequently in a collection of graphs.  This 
is an active area of research with wide applicability to fields such as 
molecular biology, information retrieval, image classification, and many 
others.  However, the problem is at its core a subgraph isomorphism problem
which is proven to be NP-Complete\cite{cook1971complexity}.  This helps make
the problem particularly well suited to large, supercomputer class systems.

We intend to modify an existing frequent subgraph mining algorithm to exploit
parallelism using the MPI framework. The hope is that the search space can
be divided up among multiple nodes within a cluster to provide faster overall
search times.  While attempts have been made to parallelize some FSM 
algorithms in the past\cite{buehrer2005parallel,gspancuda}, these 
approaches either use GPUs or limit themselves to shared memory machines. 
To our knowledge this would be the first FSM system implemented with MPI.

\subsection{Applications}
\label{subsec:applications}

Many problems rely on looking for common patterns in substructures to 
identify similarities between related substances.  The main area 
for this is protein structures and toxicology discovery, as both 
protein structures\cite{substructures} and chemicals\cite{toxic} can be easily represented as 
graphs. Other applications include video indexing and compression\cite{videosub}, 
image classification\cite{plagram}, storage compression\cite{stored}, and web link 
indexing\cite{freqtrees}.

\subsection{Definition}
\label{subsec:complexity}

A labelled graph $G'$ is a {\bf subgraph} of $G$ if
\begin{itemize}
	\item{All vertices of $G'$ are in $G$}
	\item{For each vertex in both graphs, the vertex labels are the same}
	\item{All edges of $G'$ are in $G$ }
	\item{For each edge in both graphs, the vertex labels are the same}
\end{itemize}

A graph is a {\bf frequent subgraph} of a {\bf graph database} if the 
subgraph appears in $\sigma$ or more graphs within the database.


