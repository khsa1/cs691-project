\section{Implementation}
\label{sec:implementation}

Our code implements the gSpan algorithm in C. We begin by reading in
the graphs from the input file. The database of graphs is
stored as a doubly-linked-list using the GLib GList structure.

The nodes in the database are sorted by their frequency. The second input
into our code is the minimum support. We prune all infrequent nodes and
edges that do not satisfy this minimum support. The list of subgraphs is
intially defined to be all frequent 1-edge graphs in the database. 

This list of subgraphs is stored on each process. This list of 1-edge graphs
is then evenly distribued among the processes. For each of these edges,
subgraph mining is recursively called to grow the subgraphs in order to find
all frequenct descendants. The search stops when the support is less than the
minimum support or the subgraph and its descendants have already been
discovered.

After creating the local list of subgraphs. We iterate through each process,
printing the id, support, vertices, and edges of each subgraph.
Message passing is required to correctly compute the correct global 
id of each subgraph.
