\section{Implementation}
\label{sec:implementation}

Our code implements the gSpan algorithm in C. We begin by reading in
the graphs from the input file. The database of graphs is
stored as a doubly-linked-list using the GLib GList structure.

The nodes in the database are sorted by their frequency. The second input
into our code is the minimum support, $\sigma$. We prune all infrequent nodes
and edges that do not satisfy this minimum support. The list of subgraphs is
initially defined to be all frequent 1-edge graphs in the database.  This list
of subgraphs is stored on each process.

In our first implementation, the list of 1-edge graphs is evenly distributed
among the processes. Each of these edges is mined by recursively calling the
subgraph mining subprocedure to grow the subgraphs in order to find all
frequent descendants.
The search stops when the support is less than the
minimum support or the subgraph and its descendants have already been
discovered.

Unlike the first implementation in which all edges are statically allocated
to each process, the second implementation uses centralized dynamic
load balancing to allocate edges to processes.
In this method, process 0 acts as the master
process which contains a counter with the next edge that needs to be mined.
All other processes are workers that mine their edge. Once a worker process
finishes mining its edge it will send a message with its rank to process 0.
Process 0 will then send back the id of the next edge that needs to be mined
and will increment the counter. Once the edge id of each process is greater
than the total number of edges for each process we know that all edges
have been mined. 

In both implementations, after creating the local list of subgraphs we
iterate through each process, printing the id, support, vertices, and
edges of each subgraph.
Message passing is required to correctly compute the correct global 
id of each subgraph.
