\section{Conclusions and Future Work}
\label{sec:conclusions}

\subsection{Comparison to other implementation}
\label{subsec:comparison}

\subsection{Conclusion}
\label{subsec:conclusions}

\subsection{Future Work}
\label{subsec:future}

Our current implementation is a pure MPI implementation. In the future we
hope to develop a hybrid MPI-OpenMP implementation. This implementation
would use a combination of MPI for inter-node communcation and OpenMP
for intra-node, shared memory communcation.

In our current implementation, the entire database is stored on each node.
We are interested in developing a distributed structure for the database that
would enable us to solve problems that require more memory than is available
on a single node. Our database is currently store in a doubly-linked-list, so
passing subgraphs between MPI processes is non-trivial. Since
linked-lists are not stored in contiguous memory and the size of these
linked-lists are not known, this would most likely require the use of arrays
to pass data between MPI processes.

Finally, we would be interested in a dynamic queueing mechanism to farm
the work of child candidate graph generation out to parallel nodes. Since
the subgraph mining subprocedure is recursive and we do not know how many
calls are necessary for completion at the beginning of the algorithm it is
possible that one process can take much longer than other processes to
complete. A dynamic queueing mechanism would avoid this problem by giving work
to idling processes.
