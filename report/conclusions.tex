\section{Conclusions and Future Work}
\label{sec:conclusions}

\subsection{Comparison to other implementations}
\label{subsec:comparison}

Compared to other parallel implementations, our implementations would need
further refinement to achieve similar efficiency.  Buehrer, et al. show
much higher speedup on a synthetic dataset of ~26 (for 32 nodes on a shared
memory system), while we show a speedup of ~3.  As we do not have the exact
same dataset to test on, apples-to-apples comparison is difficult, but such a
large difference would indicate much greater speedups in general compared to
our implementation.   

The main reason for this is that we are parallelizing the mining of each 
initial edge, and given the nature of depth first search, mining one edge 
can take the majority of the runtime.  If we implemented a similar 
queue-based parallelization of the mining itself, we should achieve similar 
results to Buehrer.  Alternatively, we could use MPI to distribute the 
initial load, and use Buehrer's technique internal to each node.  

\subsection{Conclusion}
\label{subsec:conclusions}

We observed in Section~\ref{sec:results} that the performance of our method
is highly dependent on the structure of the dataset that is mined.

Generally, reasonable speedup was observed on small numbers of processes. For
the synthetic dataset we observed a reduction of runtime from 76.95~seconds
in serial to 51.89~seconds with two processes, a speedup of 1.483. A similar
speedup of 1.348 was observed for the Compound\_422 dataset. The MCG-7 dataset
demonstrates the worst-case behavior of our method. The serial runtime of
2503.03~seconds can only be reduced to 2272.73~seconds, a speedup of just
1.101. As discussed in Section~\ref{sec:results}, since our method
parallelizes the gspan algorithm by splitting up the edges between processes
but does not parallelize the mining of each individual edge our method cannot
perform faster than the time it takes to mine the slowest edge. So, if we
use our method on a dataset such as MCG-7 in which one edge takes much longer
than all other edges we observe poor performance.

Since each edge takes an indeterminate amount of time to mine, we developed
a dynamic implementation of our method. This produces slightly better results
for datasets such as the synthetic dataset where edges take approximately
the same amount of time to mine since the work is more evenly split between
processes, however we will still observe poor performance for datasets such as
MCG-7 in which one edge takes much longer to mine than all other edges. Thus,
other method of parallelization are necessary to improve parallel performance
by parallelizing the actual mining of individual edges rather than splitting
edges between processes.

\subsection{Future Work}
\label{subsec:future}

Our current implementation is a pure MPI implementation. In the future we
hope to develop a hybrid MPI-OpenMP implementation. This implementation
would use a combination of MPI for inter-node communication and OpenMP
for intra-node, shared memory communication.

In our current implementation, the entire database is stored on each node.
We are interested in developing a distributed structure for the database that
would enable us to solve problems that require more memory than is available
on a single node. Our database is currently store in a doubly-linked-list, so
passing subgraphs between MPI processes is non-trivial. Since
linked-lists are not stored in contiguous memory and the size of these
linked-lists are not known, this would most likely require the use of arrays
to pass data between MPI processes.
The benefit of this method, however, would be significant. Since our current
methods cannot perform faster than it takes to mine the slowest edge but this
method would parallelize the mining of an individual edge. By parallelizing
the work of mining an edge it would be possible to get significantly better
performance.
