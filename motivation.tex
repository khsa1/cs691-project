\section{Motivation}
\label{sec:motivation}

Frequent Subgraph Mining (FSM) is the general computational problem of 
identifying what subgraphs occur frequently in a collection of graphs.  This 
is an active area of research with wide applicability to fields such as 
molecular biology, information retrieval, image classification, and many 
others.  However, the problem is at its core a subgraph isomorphism problem
which is proven to be NP-Complete\cite{cook1971complexity}.  This helps make
the problem particularly well suited to large, supercomputer class systems.

We intend to modify an existing frequent subgraph mining algorithm to exploit
parallelism using the MPI framework. The hope is that the search space can
be divided up among multiple nodes within a cluster to provide faster overall
search times.  While attempts have been made to parallelize some FSM 
algorithms in the past\cite{buehrer2005parallel,gspancuda}, these 
approaches either use GPUs or limit themselves to shared memory machines. 
To our knowledge this would be the first FSM system implemented with MPI.

